% Encoding und Sprache -------------------------------------------------------
\usepackage[utf8]{inputenc}
\usepackage[ngerman]{babel}
\usepackage[T1]{fontenc} % z.B. Umlaute auch als solche Drucken
\usepackage{textcomp} % Euro-Zeichen etc.
% \usepackage{relsize} % Schriftgröße relativ festlegen
\usepackage{mathptmx} % gut lesbare Schriftart
\usepackage{lmodern} % Schriftart bleibt auch bei fettgedruckten Wörtern schön

% Tabellen --------------------------------------------------------------------
% \usepackage{tabularx}
% für lange Tabellen
% \usepackage{longtable}
% \usepackage{array}
% \usepackage{ragged2e}
% \usepackage{lscape}

% Graphiken -------------------------------------------------------------------
\usepackage{graphicx}
\usepackage{graphics}
% \usepackage{floatflt} % zum Umfließen von Bildern

% sonstiges -------------------------------------------------------------------
% \usepackage{amsmath,amsfonts} % Befehle aus AMSTeX für mathematische Symbole
% \usepackage{enumitem} % anpassbare Enumerates/Itemizes
\usepackage{xspace} % entscheidet, ob Leerzeichen hinter Makro oder nicht

% \usepackage{makeidx} % für Index-Ausgabe mit \printindex
\usepackage[printonlyused]{acronym} % Akronyme: es werden nur benutzte Definitionen aufgelistet

\usepackage{setspace}
\usepackage{geometry}

% Symbolverzeichnis
\usepackage[intoc]{nomencl}
\let\abbrev\nomenclature
\setlength{\nomlabelwidth}{.25\hsize}
\renewcommand{\nomlabel}[1]{#1 \dotfill}
\setlength{\nomitemsep}{-\parsep}

\usepackage{varioref} % Elegantere Verweise. „auf der nächsten Seite“
\usepackage{chngcntr} % fortlaufendes Durchnummerieren der Fußnoten
\usepackage{ifthen} % bei der Definition eigener Befehle benötigt
\setlength{\marginparwidth }{2cm} % Fehlermeldung vermeiden
\usepackage{todonotes} % definiert u.a. die Befehle \todo und \listoftodos
\usepackage{natbib} % wichtig für korrekte Zitierweise
\bibpunct[: ]{(}{)}{;}{a}{ }{,}

\usepackage{xcolor}
\definecolor{Blau}{rgb}{0, 0.28, 0.56}


% \usepackage{pdfpages}
% \pdfminorversion=5 % erlaubt das Einfügen von pdf-Dateien bis Version 1.7, ohne eine Fehlermeldung zu werfen (keine Garantie für fehlerfreies Einbetten!)

\usepackage[
bookmarks,
bookmarksnumbered,
bookmarksopen=true,
bookmarksopenlevel=1,
colorlinks=true,
% diese Farbdefinitionen zeichnen Links im PDF farblich aus
linkcolor=Blau, % einfache interne Verknüpfungen
anchorcolor=Blau,% Ankertext
citecolor=Blau, % Verweise auf Literaturverzeichniseinträge im Text
filecolor=Blau, % Verknüpfungen, die lokale Dateien öffnen
menucolor=Blau, % Acrobat-Menüpunkte
urlcolor=Blau,
% diese Farbdefinitionen sollten für den Druck verwendet werden (alles schwarz)
%linkcolor=black, % einfache interne Verknüpfungen
%anchorcolor=black, % Ankertext
%citecolor=black, % Verweise auf Literaturverzeichniseinträge im Text
%filecolor=black, % Verknüpfungen, die lokale Dateien öffnen
%menucolor=black, % Acrobat-Menüpunkte
%urlcolor=black,
%
%backref, % Quellen werden zurück auf ihre Zitate verlinkt
pdftex,
plainpages=false, % zur korrekten Erstellung der Bookmarks
pdfpagelabels=true, % zur korrekten Erstellung der Bookmarks
hypertexnames=false, % zur korrekten Erstellung der Bookmarks
linktocpage % Seitenzahlen anstatt Text im Inhaltsverzeichnis verlinken
]{hyperref}



% zum Einbinden von Programmcode -----------------------------------------------
\usepackage{listings}
\definecolor{hellgelb}{rgb}{1,1,0.9}
\definecolor{colKeys}{rgb}{0,0,1}
\definecolor{colIdentifier}{rgb}{0,0,0}
\definecolor{colComments}{rgb}{0,0.5,0}
\definecolor{colString}{rgb}{1,0,0}
\lstset{
    float=hbp,
    basicstyle=\footnotesize\ttfamily,
    identifierstyle=\color{colIdentifier},
    keywordstyle=\color{colKeys},
    stringstyle=\color{colString},
    commentstyle=\color{colComments},
    backgroundcolor=\color{hellgelb},
    columns=flexible,
    tabsize=2,
    frame=single,
    extendedchars=true,
    showspaces=false,
    showstringspaces=false,
    numbers=left,
    numberstyle=\tiny,
    breaklines=true,
    breakautoindent=true,
    captionpos=b,
}
\lstdefinelanguage{cs}{
    sensitive=false,
    morecomment=[l]{//},
    morecomment=[s]{/*}{*/},
    morestring=[b]",
    morekeywords={
        abstract,event,new,struct,as,explicit,null,switch
        base,extern,object,this,bool,false,operator,throw,
        break,finally,out,true,byte,fixed,override,try,
        case,float,params,typeof,catch,for,private,uint,
        char,foreach,protected,ulong,checked,goto,public,unchecked,
        class,if,readonly,unsafe,const,implicit,ref,ushort,
        continue,in,return,using,decimal,int,sbyte,virtual,
        default,interface,sealed,volatile,delegate,internal,short,void,
        do,is,sizeof,while,double,lock,stackalloc,
        else,long,static,enum,namespace,string},
}
\lstdefinelanguage{php}{
    sensitive=false,
    morecomment=[l]{/*},
    morestring=[b]",
    morestring=[b]',
    alsodigit={-,*},
    morekeywords={
        abstract,and,array,as,break,case,catch,cfunction,class,clone,const,
        continue,declare,default,do,else,elseif,enddeclare,endfor,endforeach,
        endif,endswitch,endwhile,extends,final,for,foreach,function,global,
        goto,if,implements,interface,instanceof,namespace,new,old_function,or,
        private,protected,public,static,switch,throw,try,use,var,while,xor
        die,echo,empty,exit,eval,include,include_once,isset,list,require,
        require_once,return,print,unset},
}

% \usepackage{titlesec}
\usepackage{scrhack} % Fehlermeldung vermeiden
\usepackage[
%   automark, % Kapitelangaben in Kopfzeile automatisch erstellen
    headsepline, % Trennlinie unter Kopfzeile
    ilines, % Trennlinie linksbündig ausrichten
    singlespacing=true
]{scrlayer-scrpage}

% Metadaten für das PDF-Dokument
\hypersetup{
    pdftitle={\titel{} -- \untertitel},
    pdfauthor={\autor},
    pdfcreator={\autor},
    pdfsubject={\titel{} -- \untertitel},
    pdfkeywords={\titel{} -- \untertitel},
}
