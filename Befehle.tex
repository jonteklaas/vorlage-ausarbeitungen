% Abkürzungen, ggfs. mit korrektem Leerraum
\newcommand{\bs}{$\backslash$\xspace}
\newcommand{\bspw}{bspw.\xspace}
\newcommand{\bzw}{bzw.\xspace}
\newcommand{\ca}{ca.\xspace}
\newcommand{\dahe}{\mbox{d.\,h.}\xspace}
\newcommand{\etc}{etc.\xspace}
\newcommand{\eur}[1]{\mbox{#1\,\texteuro}\xspace}
\newcommand{\evtl}{evtl.\xspace}
\newcommand{\ggfs}{ggfs.\xspace}
\newcommand{\Ggfs}{Ggfs.\xspace}
\newcommand{\gqq}[1]{\glqq{}#1\grqq{}}
\newcommand{\inkl}{inkl.\xspace}
\newcommand{\insb}{insb.\xspace}
\newcommand{\ua}{\mbox{u.\,a.}\xspace}
\newcommand{\usw}{usw.\xspace}
\newcommand{\Vgl}{Vgl.\xspace}
\newcommand{\vgl}{vgl.\xspace}
\newcommand{\zB}{\mbox{z.\,B.}\xspace}

% Befehle für häufig anfallende Aufgaben
\newcommand{\Abbildung}[1]{\autoref{fig:#1}}
\newcommand{\Anhang}[1]{\appendixname{}~\ref{#1}: \nameref{#1} \vpageref{#1}}
\newcommand{\includegraphicsKeepAspectRatio}[2]{\includegraphics[width=#2\textwidth,height=#2\textheight,keepaspectratio]{#1}}
\newcommand{\includegraphicsRotateAndKeepAspectRatio}[2]{\includegraphics[width=#2\textwidth,height=#2\textheight,keepaspectratio,angle=90,origin=c]{#1}}
% \newcommand{\Zitat}[2][\empty]{\ifthenelse{\equal{#1}{\empty}}{\citep{#2}}{\citep[#1]{#2}}}
\newcommand{\Autor}[1]{\textsc{#1}} % zum Ausgeben von Autoren
\newcommand{\itemd}[2]{\item{\textbf{#1}}\\{#2}} % erzeugt ein Listenelement mit fetter Überschrift
\newcommand{\Abschnitt}[1]{Abschnitt~\ref{sec:#1}~(\nameref{sec:#1})}
\newcommand{\Kapitel}[1]{Kapitel~\ref{sec:#1}~(\nameref{sec:#1})}

% verschiedene Befehle um Wörter semantisch auszuzeichnen ----------------------
\newcommand{\Index}[2][\empty]{\ifthenelse{\equal{#1}{\empty}}{\index{#2}#2}{\index{#1}#2}}
\newcommand{\Fachbegriff}[2][\empty]{\ifthenelse{\equal{#1}{\empty}}{\textit{\Index{#2}}}{\textit{\Index[#1]{#2}}}}
\newcommand{\NeuerBegriff}[2][\empty]{\ifthenelse{\equal{#1}{\empty}}{\textbf{\Index{#2}}}{\textbf{\Index[#1]{#2}}}}