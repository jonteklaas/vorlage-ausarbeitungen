% Seitenränder -----------------------------------------------------------------
\setlength{\topskip}{\ht\strutbox} % behebt Warnung von geometry
\geometry{a4paper,left=25mm,right=25mm,top=25mm,bottom=20mm,includefoot,footskip=15mm} % bei Druck left=40mm

% Kopfzeile
\renewcommand{\headfont}{\normalfont} % Schriftform der Kopfzeile
\ihead{\large{\textsc{\titel}}\\ \small{\untertitel} \\[2ex] \textit{\ihead}} % Titel, Untertitel und Trennstrich in Kopfzeile
% \ihead{\large{\textsc{\titel}}\\ \small{\untertitel} \\[2ex] \textit{\headmark}} % Titel, Untertitel, Kapitel und Trennstrich in Kopfzeile
% Dazu muss die dadrüberliegende Zeile kommentiert werden und im usepackage-Befehl von {scrlayer-scrpage} in Packages das Automark auskommentiert werden
\chead{} % nichts mittig in Kopfzeile
\ohead{\includegraphics[scale=0.3]{\logo}} % Logo rechts in Kopfzeile
\setlength{\headheight}{20mm} % Höhe der Kopfzeile

% Fußzeile
\ifoot{\autor, \jahr}
\cfoot{}
\ofoot{\pagemark}

\spacing{1.3} % Zeilenabstand 1,3 Standard
\frenchspacing % erzeugt ein wenig mehr Platz hinter einem Punkt

% Schusterjungen und Hurenkinder vermeiden
\clubpenalty = 10000
\widowpenalty = 10000
\displaywidowpenalty = 10000

\counterwithout{footnote}{section} % Fußnoten fortlaufend durchnummerieren
\setcounter{tocdepth}{3} % im Inhaltsverzeichnis werden die Kapitel bis zum Level der subsubsection übernommen (eigentlich mit \subsubsectionlevel anstatt 3)
\setcounter{secnumdepth}{3} % Kapitel bis zum Level der subsubsection werden nummeriert (eigentlich mit \subsubsectionlevel anstatt 3)

% Aufzählungen anpassen
\renewcommand{\labelenumi}{\arabic{enumi}.}
\renewcommand{\labelenumii}{\arabic{enumi}.\arabic{enumii}.}
\renewcommand{\labelenumiii}{\arabic{enumi}.\arabic{enumii}.\arabic{enumiii}}

% Tabellenfärbung:
\definecolor{heading}{rgb}{0.64,0.78,0.86}
\definecolor{odd}{rgb}{0.9,0.9,0.9}

% auch subsubsection nummerieren
\setcounter{secnumdepth}{3}
\setcounter{tocdepth}{3}